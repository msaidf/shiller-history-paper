\documentclass[a4paper, 12pt]{article}

\usepackage[english]{babel}
\usepackage[utf8]{inputenc}
\usepackage{amsmath}
\usepackage{graphicx}
\usepackage{setspace}
\usepackage{natbib}
\bibliographystyle{apalike}

\title{Robert J. Shiller}
\author{Muhamad Said Fathurrohman}
\date{April, 2014}

\begin{document}

\maketitle

\doublespace

\section{Introduction}
Robert James Shiller is a Professor of Economics at Yale University. He received the 2013 Nobel Memorial Prize in Economic Sciences jointly with Eugene Fama and Lars Peter Hansen for their empirical analysis of asset prices. He was born in Detroit, Michigan on March 29, 1946. He earned his B.A. degree from University of Michigan in 1967. Then, he went to MIT where he received his S.M. degree in 1968 and Ph.D. degree in 1972. His wrote dissertation thesis entitled ``Rational Expectations and the Structure of Interest Rates'' under the supervision of Franco Modigliani. He started teaching at Yale in 1982.  

His works are mostly in the topics of financial economics. He started from questioning the anomaly of asset markets that seem behaving not in a way consistent with the dominant view that time, which is based on rationality assumption. Later he looked for explanation from newly developed behavioral economics. He then broadened his research to apply behavioral approach in macroeconomic field, in conjunction with George Akerlof who has also received nobel prize.   

Shiller's role in advancing behavioral finance is at least in his successes in empirically showing that the neoclassical finance theory cannot explain real-world finance, and in his highly accurate prediction of bubble in asset markets that can only happen if market participants behave as the way behavioral finance describe them. He is not known as much by his theoretical breakthrough as his convincing empirical evidence in either refuting mainstream thought and supporting his prediction.   

\section{Challenging Efficient Market Hypothesis}

The dominant view of financial economics from the late of 1960's is Efficient Market Hypothesis brought by \citet{fama1965behavior} that assumes that stock market price reflects rational decision by market participants that already take into account all available information. The consequence of this hypothesis is that the market price has little predictability since all of the predictable part of price movement has been captured by the action of buying and selling by market participants, hence only random movement drives the short run fluctuation of asset prices. One consequence of this is that one cannot beat the market all the time, and there is no point of doing share analysis. People will be better off holding a broad portfolio of shares to minimize risks. 

Another important consequence is that there is no market bubble, a large deviation of market prices from what the fundamental factors indicated. Any hike and slump in asset prices must represent the new information that projects the future returns.  

Shiller challenges this view started by his empirical paper \citep{shiller1981stock} that shows that the asset prices volatility cannot be justified by the subsequent changes in dividends for efficient market hypothesis to hold. The asset returns highly fluctuates around its discounted value of subsequently realized dividends. The deviations are so large that it is impossible to attribute them to such things as data errors, tax changes, or price index problem. 

In explaining this anomaly, however, in this paper Shiller had not brought up behavioral explanation yet. He just offered some tweaks on the existing rational model such as changes in expected real interest rates or in market uncertainty.

His subsequent works goes further by showing that asset prices can actually be predicted in the long run, although it could not be predicted in the short run. He himself set up some indicators that serves to guide market how far they are from fundamental. When asset prices moved away from what this fundamental factors predicted, then it must be a bubble.

It is Shiller's work in explaining asset price movement, not in his works in behavioral economics and finance, which cause him earned the nobel prize. Taken together, both Fama and Shiller findings explain the behavior of asset prices in the short run and in the long run, respectively. That is why Lars Hansen also received the prize along with them, since he made important methodological contribution that advance research in asset prices. The 2013 Nobel prize was not an announcement which finance school of thought is ruling. 

\section{Irrational Bubble}

Although the 1981 paper was the cornerstone of his body of works that make him received the Nobel prize, his work that is highly influential to the real world financial market is his book ``Irrational Exuberance'' that warned financial market about the asset prices had been too high to be justified by the asset returns. The first edition of this book that warned about bubble happening in stock market was published in March 2000, just before the stock market crashed in the following year. The second edition \citep{shiller2005irrational} extended the first edition by looking at the developing bubble in the housing market, which then also crashed two years later.

This book talks mainly about bubble and how financial market keeps producing it. Shiller's thought in this book can be summarized by how he defined bubble as following.
\begin{quote}
	A situation in which news of price increases spurs investor enthusiasm which spreads by psychological contagion from person to person, in the process amplifying stories that might  justify the price increase and bringing in a larger and larger class of investors, who, despite  doubts about the real value of the investment, are drawn to it partly through envy of others’  successes and partly through a gambler’s excitement \citet{shiller2005irrational}.
\end{quote}

This definition is a rebel beyond the efficient market view, but also neoclassical economic thought in general. It views that investor decisions under bubble condition is not driven by their rational calculation, but instead by emotional aspects such as enthusiasm, doubts, envy, and excitement. The view of bubble as a product of irrational behavior is in contrast to the rational bubble theory which is more familiar in macroeconomics. 

This view is closer to the old-Keynesian view. Indeed, Shiller likes to describe financial market using Keynes' metaphor ``beauty contest'', where people will  win if they choose the face that is thought as the prettiest by most contest participants. 

The way Shiller describe the process of bubble formation in asset market. First, he explain market fluctuations as caused by precipitating factors such as politics, technology, and demography. This fluctuations then sometimes amplified by some mechanisms, that set up the stage of spiral of confidence and price changes, that can go either upward or downward.

The news media has important roles in the amplification of market fluctuation by amplifying stories related to investment. The particular stories that set up a bubble is the new era theories, that ``this time is different'' as the way \citet{reinhart2009this} put it.

He argues that public cannot compute the true value of the markets as defined by economic and financial theory, instead they rely more on psychological anchors. Shiller combines both social psychology and socialogy to explain why so many different people can change their opinions at the same time. 

Shiller takes further his view about how confidence in market plays dominant role by actually creating an indices measuring the confidence of stock market. He does not infer the confidence from economic data as economists usually do it. Instead, he asks directly investors what they thought. His survey has encouraged similar surveys undertaken in Japan and China.

\section{Good Finance}

Despite his view of financial market often behave irrationally, Shiller has a good faith on financial market and financial innovation. He argues that financial innovation has increased general well-being and more innovations are needed, but not in the way that has been done recently and has created the largest global crisis ever seen after Great Depression.

In order to improve the role of financial system in society, he asserts that academic finance need to broaden its interest over designing optimal portfolios of investments. Shiller view finance as the science of structuring economic arrangements and of the stewardship of the assets necessary to achieve a set of goals. He views finance as not only about making money, but it is a device to achieve goals by a wide range of society members. Finance help people achieving that goal by aligning people's incentive so that they can cooperate \citep{shiller2013finance}.

Shiller sees that current financial system actually rebut Marx idea that capitalism take away social means of subsistence and production to a handful of capitalist. Anybody can now become a producer because they can get the capital they need from financial market. Capitalist is no longer a class that only certain people can become part of it. 

In developing new financial instruments that can serve society well, he promotes the use of behavioral analysis to help people decide on the best option for them. The imperfect rationality and information implies that people needs help on getting the right decision for themselves. Altering the economic benefit and cost often does not work because of this imperfection in decision making. Behavioral economics have policy suggestions that is less costly by improving the decision making process. \citet{shiller2005behavioral} mentions some of institutional innovation that improves society well-being by means that is justified by behavioral economics, but not so by neoclassical economics. Social insurance is justified by behavioral economics since people tend to be overconfidence and underestimate the low probability risks. Even if mandatory program is considered too restrict individual liberty, making participation in program as a default option can still help many people involved in the program that is good for them.

\section{Behavioral Macroeconomics}

Moving beyond finance, Shiller sees that behavioral approach can also help explaining macroeconomic phenomenon. He described his view about the impact of psychological factors on macroeconomic in his book coauthoring with George Akerlof \citep{akerlof2010animal}. Shiller does not think that macroeconomic cannot be explained by rationality framework or just a bit deviation from it.  Instead, he argues that the deviations exist in the everyday economy and has to be explained in its own framework rather than started from rationality model.

Modern macroeconomics carry Adam Smith idea that society is served the best when people pursue their self interest in free market by interpreting it that one of the good thing brought by free market economy is the stability and full employment. Any shock to economy will be brought back to the balance and full employment by market price adjustment.

Shiller still acknowledge the truth of this classical view, but he tries to revive Keynes idea on animal spirits, i.e. the irrational side of economic actors. People do not decide investment by thorough calculation of probability, risk, and returns. There is nothing known about future to be calculated. Economy also keeps changing because people change their thought pattern continuously, instead of consistently as neoclassical economics suggest. The thought pattern includes emotional aspects such as confidence, temptations, envy, resentment, illusions, and is driven by the changing stories about the nature of the economy.

\citet{akerlof2010animal} offers the extended animal spirit theory by breaking it down to five different aspects: confidence, fairness, corruption and antisocial behavior, money illusion, and stories. The first aspect of their theory, confidence, is the cornerstone feedback mechanisms between it and the economy. This feedback mechanisms is the main mechanism amplifying market fluctuation. Fairness explains much about the rigidity of wages and prices that becomes a center of debate between neoclassical and neo-Keynesian macroeconomists. Money illusion idea has been around for some time but macroeconomists did not take it seriously since it is not consistent with rationality framework, so Akerlof and Shiller revive this idea by providing more arguments. The last aspect, stories are carried from Shiller's idea that has been developed in his Irrational Exuberance book.

\section{Conclusions}

Shiller contribution in economics that makes him earned nobel prize is his empirical works that helps to understand the behavior of market price. However, this recognition really underestimates his actual contribution in refuting the most influential theory in financial economics, the efficient market hypothesis. His empirical findings, market prediction, and his thought in behavioral finance has a huge implication in the financial market regulation. Had his thought was taken seriously by the authority, the last two asset market crashes might be avoidable. Although it can be thought the other way, that actually his writing on the bubble of stock market and house market has set the drop in market confidence and cause market to crash.

The nobel has not taken into account his later contribution in behavioral finance and macroeconomics that has pioneered the development of this school of thought. He might be not the one who pioneered the development of behavioral economics as much as Kahneman, Tversky, and Thaler. However, in behavioral finance and macroeconomics, he probably deserves another noble prize for his contribution in this subject.

\bibliography{History.bib}
\end{document}
